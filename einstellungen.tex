%%%%%%%%%%%%%%%%%%%%%%%%%%%%%%%%%%%%%%%%%%%%%%%%%%%%%%%%%%%%%%%%%%%%%%%%%%%%%%%
%                                   Einstellungen
%
% Hier können alle relevanten Einstellungen für diese Arbeit gesetzt werden.
% Dazu gehören Angaben u.a. über den Autor sowie Formatierungen.
%
%
%%%%%%%%%%%%%%%%%%%%%%%%%%%%%%%%%%%%%%%%%%%%%%%%%%%%%%%%%%%%%%%%%%%%%%%%%%%%%%%


%%%%%%%%%%%%%%%%%%%%%%%%%%%%%%%%%%%% Sprache %%%%%%%%%%%%%%%%%%%%%%%%%%%%%%%%%%%
%% Aktuell sind Deutsch und Englisch unterstützt.
%% Es werden nicht nur alle vom Dokument erzeugten Texte in
%% der entsprechenden Sprache angezeigt, sondern auch weitere
%% Aspekte angepasst, wie z.B. die Anführungszeichen und
%% Datumsformate.
\setzesprache{de} % de oder en
%%%%%%%%%%%%%%%%%%%%%%%%%%%%%%%%%%%%%%%%%%%%%%%%%%%%%%%%%%%%%%%%%%%%%%%%%%%%%%%%

%%%%%%%%%%%%%%%%%%%%%%%%%%%%%%%%%%% Angaben  %%%%%%%%%%%%%%%%%%%%%%%%%%%%%%%%%%%
%% Die meisten der folgenden Daten werden auf dem
%% Deckblatt angezeigt, einige auch im weiteren Verlauf
%% des Dokuments.
\setzemartrikelnr{7866387}
\setzekurs{TIT14}
\setzetitel{Erweiterung einer 2D-Projektion mit Augmented Reality zur Darstellung eines Luftbetankungsvorgangs in einem Flugsimulator}
\setzedatumAbgabe{\DTMdate{2023-05-02}}%\today}
\setzefirma{Airbus Defence and Space}
\setzefirmenort{Manching}
\setzeabgabeort{Ingolstadt}
\setzeabschluss{Master of Science}
\setzestudiengang{Informatik}
\setzedhbw{Fakultät Informatik} %Fakultät
\setzebetreuer{Detlef Schiron}
\setzegutachter{Prof. Dr. Thomas Grauschopf}
\setzegutachterzwei{Prof. Dr. Bernd Hafenrichter}
\setzedatumAnmeldung{\DTMdate{2022-11-02}}
\setzearbeit{Masterarbeit}
\setzeautor{Nil Kuchenbäcker}
%%%%%%%%%%%%%%%%%%%%%%%%%%%%%%%%%%%%%%%%%%%%%%%%%%%%%%%%%%%%%%%%%%%%%%%%%%%%%%%%

%%%%%%%%%%%%%%%%%%%%%%%%%%%% Literaturverzeichnis %%%%%%%%%%%%%%%%%%%%%%%%%%%%%%
%% Bei Fehlern während der Verarbeitung bitte in ads/header.tex bei der
%% Einbindung des Pakets biblatex (ungefähr ab Zeile 110,
%% einmal für jede Sprache), biber in bibtex ändern.
\newcommand{\ladeliteratur}{%
\addbibresource{bibliographie.bib}
%\addbibresource{weitereDatei.bib}
}
%% Zitierstil
%% siehe: http://ctan.mirrorcatalogs.com/macros/latex/contrib/biblatex/doc/biblatex.pdf (3.3.1 Citation Styles)
%% mögliche Werte z.B numeric-comp, alphabetic, authoryear
\setzezitierstil{numeric-comp}
%%%%%%%%%%%%%%%%%%%%%%%%%%%%%%%%%%%%%%%%%%%%%%%%%%%%%%%%%%%%%%%%%%%%%%%%%%%%%%%%

%%%%%%%%%%%%%%%%%%%%%%%%%%%%%%%%% Layout %%%%%%%%%%%%%%%%%%%%%%%%%%%%%%%%%%%%%%%
%% Verschiedene Schriftarten
% laut nag Warnung: palatino obsolete, use mathpazo, helvet (option scaled=.95), courier instead
\setzeschriftart{lmodern} % palatino oder goudysans, lmodern, libertine

%% Paket um Textteile drehen zu können
%\usepackage{rotating}
%% Paket um Seite im Querformat anzuzeigen
%\usepackage{lscape}

%% Seitenränder
\setzeseitenrand{3.5cm}

%% Abstand vor Kapitelüberschriften zum oberen Seitenrand
\setzekapitelabstand{15pt}

%% Spaltenabstand
\setzespaltenabstand{10pt}
%%Zeilenabstand innerhalb einer Tabelle
\setzezeilenabstand{1.5}
%%%%%%%%%%%%%%%%%%%%%%%%%%%%%%%%%%%%%%%%%%%%%%%%%%%%%%%%%%%%%%%%%%%%%%%%%%%%%%%%

%%%%%%%%%%%%%%%%%%%%%%%%%%%%% Verschiedenes  %%%%%%%%%%%%%%%%%%%%%%%%%%%%%%%%%%%
%% Farben (Angabe in HTML-Notation mit großen Buchstaben)
\newcommand{\ladefarben}{%
    \definecolor{LinkColor}{HTML}{000000}
    \definecolor{ListingBackground}{HTML}{FCF7DE}

}
%% Mathematikpakete benutzen (Pakete aktivieren)
\usepackage{amsmath}
\usepackage{amssymb}

%% Programmiersprachen Highlighting (Listings)
\newcommand{\listingsettings}{%
    \lstset{%
        language=C++,			% Standardsprache des Quellcodes
        numbers=left,			% Zeilennummern links
        stepnumber=1,			% Jede Zeile nummerieren.
        numbersep=5pt,			% 5pt Abstand zum Quellcode
        numberstyle=\tiny,		% Zeichengrösse 'tiny' für die Nummern.
        breaklines=true,		% Zeilen umbrechen wenn notwendig.
        breakautoindent=true,	% Nach dem Zeilenumbruch Zeile einrücken.
        postbreak=\space,		% Bei Leerzeichen umbrechen.
        tabsize=2,				% Tabulatorgrösse 2
        basicstyle=\ttfamily\footnotesize, % Nichtproportionale Schrift, klein für den Quellcode
        showspaces=false,		% Leerzeichen nicht anzeigen.
        showstringspaces=false,	% Leerzeichen auch in Strings ('') nicht anzeigen.
        extendedchars=true,		% Alle Zeichen vom Latin1 Zeichensatz anzeigen.
        captionpos=b,			% sets the caption-position to bottom
        backgroundcolor=\color{ListingBackground}, % Hintergrundfarbe des Quellcodes setzen.
        xleftmargin=0pt,		% Rand links
        xrightmargin=0pt,		% Rand rechts
        frame=single,			% Rahmen an
        frameround=ffff,
        rulecolor=\color{darkgray},	% Rahmenfarbe
        fillcolor=\color{ListingBackground},
        keywordstyle=\color[rgb]{0.133,0.133,0.6},
        commentstyle=\color[rgb]{0.133,0.545,0.133},
        stringstyle=\color[rgb]{0.627,0.126,0.941},
        escapechar=\^
    }
}
%%%%%%%%%%%%%%%%%%%%%%%%%%%%%%%%%%%%%%%%%%%%%%%%%%%%%%%%%%%%%%%%%%%%%%%%%%%%%%%%

%%%%%%%%%%%%%%%%%%%%%%%%%%%%%%%% Eigenes %%%%%%%%%%%%%%%%%%%%%%%%%%%%%%%%%%%%%%%
%% Hier können Ergänzungen zur Präambel vorgenommen werden (eigene Pakete, Einstellungen)

%Fuß und Kopfzeile
\usepackage{scrlayer-scrpage}
%\ifoot{\autor} %Ich, links unten
%\cfoot{}
%\ofoot{\thepage}
%\ihead{\leftmark}
%\ohead{\leftmark}
%\automark[]{section}

%\setheadsepline{1pt} %Dicke der Trennlinie Kopfzeile - Text
%\setfootsepline{0.5pt} %Dicke der Trennlinie Fusszeile - Text
%\pagestyle{scrheadings} %gemachte Einstellungen anwenden

\usepackage{pdfpages}
\usepackage{rotating}
\usepackage{svg}
\usepackage{booktabs}
% \usepackage[table,xcdraw]{xcolor}
\usepackage{tabularx}
\usepackage{ltablex}
% \usepackage{todonotes}
\usepackage[disable]{todonotes}
\usepackage{siunitx}
\usepackage{adjustbox}
\iflang{de}{
    \usepackage[ngerman]{datetime2} % für \DTMdiplaydate{} und \DTMdate{}
}
\iflang{en}{
    \usepackage[en-GB]{datetime2} % für \DTMdiplaydate{} und \DTMdate{}
}

% \svgsetup{
%     inkscapeexe="C:/Program Files/Inkscape/bin/inkscape.exe",
%     inkscapepath=svgsubdir
% }

\definecolor{hellgelb}{rgb}{1,1,0.8}
\definecolor{colKeys}{rgb}{0,0,1}
\definecolor{colIdentifier}{rgb}{1,0,0}
\definecolor{colComments}{rgb}{0,0.7,0.4}
\definecolor{colString}{rgb}{0,0.5,0}
\definecolor{mygray}{rgb}{0.5,0.5,0.5}

% \makeatletter
% \renewcommand*\env@matrix[1][\arraystretch]{%
%   \edef\arraystretch{#1}%
%   \hskip -\arraycolsep
%   \let\@ifnextchar\new@ifnextchar
%   \array{*\c@MaxMatrixCols c}}
% \makeatother

% \usepackage[utf8]{inputenc}
% \usepackage{pgfplots}
% \DeclareUnicodeCharacter{2212}{−}
% \usepgfplotslibrary{groupplots,dateplot}
% \usetikzlibrary{patterns,shapes.arrows}
% \pgfplotsset{compat=newest}