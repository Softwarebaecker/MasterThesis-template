%!TEX root = ../dokumentation.tex

%
% vorher in Konsole folgendes aufrufen:
%	makeglossaries makeglossaries dokumentation.acn && makeglossaries dokumentation.glo
%

%
% Glossareintraege --> referenz, name, beschreibung
% Aufruf mit \gls{...}
%
% \newglossaryentry{Glossareintrag}{name={Glossareintrag},plural={Glossareinträge},description={Ein Glossar beschreibt verschiedenste Dinge in kurzen Worten}}

\newglossaryentry{Jitter}{name={Jitter}, description={In Simulationen wird als Jitter das zeitliche Zittern einer Entität im Takt bezeichnet}}

\newglossaryentry{SIL}{name={SIL}, description={SIL ist ein Protokoll zur Kommunikation zwischen Simulatoren. Es ist nicht bekannt, wofür der Name steht}}

\newglossaryentry{SpMappiong}{name={Spatial Mapping}, description={Spatial Mapping scannt die Umgebung und erstellt eine 3D-Map\cite{Teruggi.2022}}}

\newglossaryentry{Ownship}{name={Ownship}, description={Der Begriff Ownship bezeichnet das eigene Flugzeug in einem Trainings-Simulator}}

\newglossaryentry{Csharp}{name={C\#}, description={C\# ist eine objektorientierte Programmiersprache, die von Microsoft entwickelt wird. Sie wird auch häufig als C-Sharp ausgeschrieben}}

\newglossaryentry{akkommodation}{name={Akkommodation}, description={Akkommodation ist die Fähigkeit des Auges, durch Veränderung der Brechkraft der Augenlinse, Gegenstände in unterschiedlichen Entfernungen scharf zu stellen~\cite{Bergua.2017b}}}

\newglossaryentry{augmentiert}{name={augmentiert}, description={virtuell erweitert}}

\newglossaryentry{cgf}{name={CGF}, description={Ein CGF ist eine vom Computer gesteuerte Entität in der Simulation}}

\newglossaryentry{IOT}{name={Inside-Out-Tracking}, description={Als Inside-Out-Tracking wird das Tracken mithilfe einer Kamera bezeichnet, die direkt auf oder in einem \ac{MR} Headset integriert ist~\cite[p.~7]{Gourlay.2017}}}

\newglossaryentry{Shader}{name={Shader}, description={Shader sind Softwarekomponenten, die für die Verarbeitung der Grafik benutzt werden. Sie steuern das Aussehen von 3D-Modellen}}

\newglossaryentry{Vertex}{name={Vertex}, description={Ein Vertex (Mehrzahl: Vertices) ist ein Eckpunkt in einem 3D-Modell. Er kann neben der Position auch weitere Informationen, wie zum Beispiel die Farbe enthalten}}

\newglossaryentry{Konvergenz}{name={Konvergenz}, description={Die Konvergenz ist die Fähigkeit der Augen, sich auf einen nahen Punkt zu konzentrieren. Dabei drehen sich beide Augen leicht nach innen zueinander}}

% \newglossaryentry{}{}