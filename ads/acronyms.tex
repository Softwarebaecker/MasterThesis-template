%!TEX root = ../dokumentation.tex

\addchap{\langabkverz}
%nur verwendete Akronyme werden letztlich im Abkürzungsverzeichnis des Dokuments angezeigt
%Verwendung: 
%		\ac{Abk.}   --> fügt die Abkürzung ein, beim ersten Aufruf wird zusätzlich automatisch die ausgeschriebene Version davor eingefügt bzw. in einer Fußnote (hierfür muss in %		%					header.tex \usepackage[printonlyused,footnote]{acronym} stehen) dargestellt
%		\acs{Abk.}   -->  fügt die Abkürzung ein
%		\acf{Abk.}   --> fügt die Abkürzung UND die Erklärung ein
%		\acl{Abk.}   --> fügt nur die Erklärung ein
%		\acp{Abk.}  --> gibt Plural aus (angefügtes 's'); das zusätzliche 'p' funktioniert auch bei obigen Befehlen
%	siehe auch: http://golatex.de/wiki/%5Cacronym
%	\todo{Fix IG}
% \begin{acronym}[123456789][longest acronym] 
\begin{acronym}[NASA-TLX11]
    \setlength{\itemsep}{-\parsep}
    \acro{AR}{Augmented Reality}
    \acro{ARAAR}{Augmented Reality Air to Air Refueling}
    \acro{CGF}{Computer Generated Force}
    \acro{ECEF}{Earth-Centered, Earth-Fixed}
    \acro{HMD}{Head Mounted Display}
    \acro{IG}{Image Generator}
    \acrodefplural{IGs}{Image Generatoren}
    % \todo[inline]{fix Pluralanzeige}
    \acro{IR}{Infrarot}
    \acro{LCOS}{Liquid Crystal on Silicon}
    \acro{MR}{Mixed Reality}
    \acro{MRTK}{Mixed Reality Tool Kit}
    \acro{NASA-TLX}{NASA Task Load Index}
    \acro{OTW}{Out of the Window}
    \acro{PD}{Pupillendistanz}
    \acro{SSQ}{Simulator Sickness Questionnaire}
    \acro{SUS}{System Usability Scale}
    \acro{UDP}{User Datagram Protocol}
    \acro{VR}{Virtual Reality}
    \acro{WGS84}{World Geodetic System 1984}
    \acro{NED}{North East Down}
\end{acronym}

